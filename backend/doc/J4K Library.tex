\documentclass{report}
\title{Documentation J4K Java Library}
\author{Robin \bsc{Shin} et Thibaud \bsc{Lemaire}}
\date{PACT 2015-2016}

\usepackage[latin1]{inputenc}
\usepackage[T1]{fontenc}
\usepackage[francais]{babel}
\usepackage{graphicx}
\usepackage[top=2cm, bottom=2cm, left=2cm, right=3cm]{geometry}
\usepackage{verbatim}

\begin{document}
\maketitle
\chapter*{Installation}

Tout d'abord, il est \textit{imp\'eratif} d'avoir la version 1.8 du SDK Kinect et non la 2.0 car celle-ci est inexploitable (probl\`emes de pilotes).

\subsection*{T\'el\'echargement du fichier .jar}
Sur : \underline{http://research.dwi.ufl.edu/ufdw/download.php}, t\'el\'echarger ufdw.jar.

\subsubsection*{Int\'egration avec Eclipse}
\begin{enumerate}
\item Ouvrir Eclipse, aller dans Project > Properties et aller dans l'onglet Java Build Path
\item Cliquer sur le bouton "Add External JARs..." et choisir le chemin vers le fichier ufdw.jar pr\'ec\'edemment t\'el\'echarg\'e.
\item D\'eplacer le fichier "ufdw\_j4k\_32bit.dll" ou "ufdw\_j4k\_64bit.dll" en fonction de la machine utilis\'ee \`a la racine du projet Java pour \'eviter les probl\`emes de dll manquantes.
\item On peut d\'esormais importer les fichiers de la librairie gr?ce ? la commande "import edu.ufl.digitalworlds.j4k.*".
\end{enumerate}

\subsubsection*{Recommandations pour l'int\'egration avec git}
\begin{enumerate}
\item Tout d'abord, il ne faut surtout pas ignorer le fichier .classpath au risque d'avoir des probl\`emes avec git.
\item Dans le fichier .classpath, le chemin vers ufdw.jar doit ?tre relatif et non absolu au risque de conflits d\`es qu'un utilisateur souhaite faire un push ou un pull.
\end{enumerate}

\subsubsection*{Ajout d'un projet de d\'emonstration sous Eclipse}
\begin{enumerate}
\item Ouvrir Eclipse, aller dans File > Import... et s\'electionner Git > Projects from Git
\item S\'electionner URI puis cliquer sur "Next"
\item Copier \underline{http://research.dwi.ufl.edu/git/j4kdemo} dans l'espace d\'edi\'e ? l'URI et cliquer sur "Next" autant de fois que n\'ecessaire, puis "Finish" : un nouveau projet "j4kdemo" est cr\'ee.
\end{enumerate}



\chapter*{Cr\'eation de l'objet Kinect}
\section*{Initialisation}
\verb"initKinectModule();"

Initialise la Kinect, les donn\'ees, le squelette, etc...


\section*{M\'ethodes}
\verb"public void onSkeletonFrameEvent(boolean[] skeleton_tracked, float[] positions, float[] orientations,"
\newline
\verb"byte[] joint_status);"

M\'ethode appel\'ee lorsqu'un nouveau squelette est re\c{c}u. Cette m\'ethode remplace le squelette de l'attribut associ\'e de type Skeleton, cr\'ee un \'ev\'enement et l'envoie \`a tous les modules via un syst\`eme de Listeners.


\hspace{1\baselineskip}

\verb"public void onDepthFrameEvent(short[] arg0, byte[] arg1, float[] arg2, float[] arg3);"

Permet de fixer une date de lancement de la Kinect pour, entre autre, calculer le fps. Cet Event est appel\'e lorsque le depthFrame est recu.
      
      
\hspace{1\baselineskip}

\verb"public Skeleton getSkeleton();"

Getter pour r\'ecup\'erer le Skeleton.


\hspace{1\baselineskip}

\verb"public int getNumberOfDancers();"

Renvoie le nombre de personnes d\'et\'ect\'ees par la Kinect (renvoie automatiquement 1 pour l'instant).


\hspace{1\baselineskip}

\verb"public Object getVideo();"

Renvoie la vid\'eo (pas encore impl\'ement\'ee.


\hspace{1\baselineskip}

\verb"public long getFPS();"

Renvoie le nombre de squelettes re\c{c}us par seconde.



\chapter*{Comment les \'ev\'enements sont-ils r\'ecup\'er\'es?}
Les \'ev\'enements sont r\'ecup\'er\'es via un syst\`eme de Listeners.

\section*{Qu'est-ce qu'un listener?}
Un listener est une instance d'une classe qui poss�des certaines m�thodes qui sont destin�es � �tre appel�es par un gestionnaire d'�v�nement. Une classe de listener doit h\'eriter de la classe 'EventListener'. Nous allons nous servir de ces listeners pour faire les transitions entre tous les modules de notre projet.

\section*{Utilisation dans notre projet}

\subsection*{La classe KinectEvent}
Nous avons besoin de cr\'eer une nouvelle classe KinectEvent. En effet, ce sera un objet de la classe KinectEvent qui sera r\'ecup\'er\'e par les autres modules. Cette classe poss\`ede un attribut Skeleton et un attribut long repr\'esentant le temps. Elle poss\`ede \'egalement deux getters permettant de r\'ecup\'erer ces attributs, et permettant ainsi \`a la classe ayant re\c{c}u un \'ev\'enement d'exploiter le squelette.

\subsection*{L'interface KinectListenerInterface}
Cette interface va \^etre impl\'ement\'ee par toutes les classes ayant besoin d'\'ecouter la classe Kinect. Cette interface ne contient qu'une m\'ethode de signature \verb"void skeletonReceived(KinectEventInterface e);" qui va devoir \^etre impl\'ement\'ee \`a chaque fois qu'une classe impl\'emente KinectListenerInterface. C'est cette m\'ethode qui sera appel\'ee d\`es qu'un KinectEventInterface sera re\c{c}u.


\newpage
\chapter*{Comment le squelette est-il manipul\'e?}
On suppose avoir un objet de type Skeleton appel\'e squelette. Alors la classe Skeleton impl\'ement\'ee par la J4KSDK fournit directement des m\'ethodes de signatures \verb"public float get3DJointX(int joint_id);", \verb"public float get3DJointY(int joint_id);", \verb"public float get3DJointZ(int joint_id);" permettant de r\'ecup\'erer respectivement les coordonn\'ees $(x, y, z)$ du squelette. Enfin, la classe Skeleton contient plusieurs constantes de classes permettant de tracker une partie du corps bien pr\'ecise en rempla\c{c}ant l'entier \verb"joint_id" par une des constantes de classes suivantes :

\hspace{1\baselineskip}

\begin{center}
\begin{tabular}{|c|c|c|}
\hline
Identifiant & Entier associ\'e \\
\hline
SPINE\_BASE & 0 \\
\hline
SPINE\_MID & 1 \\
\hline
NECK & 2 \\
\hline
HEAD & 3 \\
\hline
SHOULDER\_LEFT & 4 \\
\hline
ELBOW\_LEFT & 5 \\
\hline
WRIST\_LEFT & 6 \\
\hline
HAND\_LEFT & 7 \\
\hline
SHOULDER\_RIGHT & 8 \\
\hline
ELBOW\_RIGHT & 9 \\
\hline
WRIST\_RIGHT & 10 \\
\hline
HAND\_RIGHT & 11 \\
\hline
HIP\_LEFT & 12 \\
\hline
KNEE\_LEFT & 13 \\
\hline
ANKLE\_LEFT & 14 \\
\hline
FOOT\_LEFT & 15 \\
\hline
HIP\_RIGHT & 16 \\
\hline
KNEE\_RIGHT & 17 \\
\hline
ANKLE\_RIGHT & 18 \\
\hline
FOOT\_RIGHT & 19 \\
\hline
SPINE\_SHOULDER & 20 \\
\hline
HAND\_TIP\_LEFT & 21 \\
\hline
THUMB\_LEFT & 22 \\
\hline
HAND\_TIP\_RIGHT & 23 \\
\hline
THUMB\_RIGHT & 24 \\
\hline
JOINT\_COUNT & 25 \\
\hline
\end{tabular}
\end{center}

\section*{Exemple pour tracker la main droite}
Il suffit de tapper les lignes de commandes : 
\newline
\verb"int x, y, z;"
\newline
\verb"x = get3DJointX(Skeleton.HAND_RIGHT);"
\newline
\verb"y = get3DJointY(Skeleton.HAND_RIGHT);"
\newline
\verb"z = get3DJointZ(Skeleton.HAND_RIGHT);"

Alors les coordonn\'ees de la main droite sont $(x, y, z)$.






\end{document}