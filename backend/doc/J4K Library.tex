\documentclass{report}
\title{Documentation J4K Java Library}
\author{Robin \bsc{Shin} et Thibaud \bsc{Lemaire}}
\date{PACT 2015-2016}

\usepackage[latin1]{inputenc}
\usepackage[T1]{fontenc}
\usepackage[francais]{babel}
\usepackage{graphicx}
\usepackage[top=2cm, bottom=2cm, left=2cm, right=3cm]{geometry}

\begin{document}
\maketitle
\chapter*{Installation}

\subsection*{T�l�chargement des fichiers .jar}
Sur : \underline{http://research.dwi.ufl.edu/ufdw/download.php}, t�l�charger ufdw.jar.

\textbf{Int�gration avec Eclipse}
\begin{enumerate}
\item Ouvrir Eclipse, aller dans Project > Properties et aller dans l'onglet Java Build Path
\item Cliquer sur le bouton "Add External JARs..." et choisir le chemin vers le fichier ufdw.jar pr�c�demment t�l�charg�.
\item On peut d�sormais importer les fichiers de la librairie gr�ce � la commande "import edu.ufl.digitalworlds.j4k.*".
\end{enumerate}

\textbf{Ajout d'un projet de d�monstration sous Eclipse}
\begin{enumerate}
\item Ouvrir Eclipse, aller dans File > Import... et s�lectionner Git > Projects from Git
\item S�lectionner URI puis cliquer sur "Next"
\item Copier \underline{http://research.dwi.ufl.edu/git/j4kdemo} dans l'espace d�di� � l'URI et cliquer sur "Next" autant de fois que n�cessaire, puis "Finish" : un nouveau projet "j4kdemo" est cr�e.
\end{enumerate}

\chapter*{Liste des m�thodes de cette biblioth�que}
\section*{Constructeurs}
\textbf{public J4KSDK(tex);}
The constructor of the J4KSDK class. It establishes connection with the native library, which uses the Microsoft's Kinect SDK. This constructor can automatically initialize the object based on the type of the Kinect sensor that  is connected. If various types of sensors are connected priority is given to Kinect 1 devices.

\hspace{1\baselineskip}

\textbf{public J4KSDK(byte kinect\_type);}
Another constructor of the J4KSDK class. This constructor instantiates the object based the type of the Kinect sensor, which is passed as argument.

\hspace{1\baselineskip}

\textbf{public J4KSDK(byte kinect\_type, int id)}
This constructor instantiates the object based the type of the Kinect sensor and id, which are passed as arguments. For example if there are two Kinect 1 sensors, the first one corresponds to id=0 and the second one to id=1


\section*{M�thodes}
\textbf{public static final byte MICROSOFT\_KINECT\_1 = 0x1;}
\textbf{public static final byte MICROSOFT\_KINECT\_2 = 0x2;}
These two constants specify the different types of Kinect sensors to be used in the following constructors.

\hspace{1\baselineskip}

\textbf{public static final int COLOR = 0x1;}
\textbf{public static final int INFRARED = 0x2;}
\textbf{public static final int LONG\_EXPOSURE\_INFRARED = 0x4;}
\textbf{public static final int DEPTH = 0x8; }
The following constants represent the different types of data streams.

\hspace{1\baselineskip}

\textbf{public static final int PLAYER\_INDEX = 0x10;}
\textbf{public static final int SKELETON = 0x20;}
PLAYER\_INDEX is a stream of 2D image frames, which contain the id of the depicted 
      player in each pixel of the depth frame.
      
\hspace{1\baselineskip}

\textbf{public static final int UV = 0x100;}
UV is a stream of 2D frames, which contain the U,V texture coordinate mapping for each pixel in the depth frame.

\hspace{1\baselineskip}

\textbf{public static final int XYZ = 0x1000;}
XYZ is a stream of 2D frames, which contain the X,Y,Z coordinates that correspond to each depth pixel in the depth frame.

\hspace{1\baselineskip}

\textbf{public int start(int flags);}
This method turns on the Kinect sensor and initializes the data streams specified by the input flags. The flags can be specified using the above types of streams according to your needs. For example flag=COLOR | DEPTH | SKELETON; initializes the color, depth, and skeleton streams.

\hspace{1\baselineskip}

\textbf{public void stop();}
This method turns off the Kinect sensor, and stops all the open streams.


\chapter*{Autres ressources disponibles sur ce site}
\section*{D'autres m�thodes}
Disponibles sur : \underline{http://research.dwi.ufl.edu/ufdw/j4k/J4KSDK.php}

\section*{Exemples de codes utilisant cette librairie}
Disponibles sur : \underline{http://research.dwi.ufl.edu/ufdw/j4k/examples.php}

\section*{Comment cr�er notre propre programme Java utilisant la Kinect?}
Tutoriel disponible sur : \underline{http://research.dwi.ufl.edu/ufdw/j4k/examples.php\#how}




\end{document}